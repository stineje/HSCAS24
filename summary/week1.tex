\documentclass[times, twocolumn, 10pt]{article}
\usepackage{url}
\usepackage{fancyhdr}
\usepackage{amsmath, amsfonts, amsthm, amssymb}  
\usepackage{epsfig}
\usepackage{lastpage}
\usepackage{hyperref}
\usepackage{graphicx}
\usepackage{color}
\usepackage{epstopdf}
\usepackage{fancyvrb}

\usepackage{geometry}
\geometry{letterpaper, left=1in, right=1in, top=1in, bottom=1in}

\pagestyle{fancy}
\fancyhf{}
\renewcommand{\headrulewidth}{0pt}
\rfoot{\thepage/\pageref{LastPage}}
%\lhead{OSU ECEN 4233 - High-Speed Computer Arithmetic - Spring 2021}
\lfoot{\LaTeX}

\begin{document}

\title{Representation of Numbers and their Algorithms}
\author{James E. Stine \\
Electrical and Computer Engineering Department\\
Oklahoma State University \\
Stillwater, OK 74078, USA}
\date{}

\maketitle
%\thispagestyle{plain}\pagestyle{plain}
\begin{enumerate}
\item Algorithms are well-defined computational procedures that takes
  some values as input and produces some value or set of values as an
  output.
\item An algorithm is said to be correct for every instance (input),
  if it halts or stops with the correct output.
  \begin{itemize}
  \item Reference: Cormen, Leiserson, Rivest, Stein,
    \underline{Introduction to Algorithms}, MIT Press
  \end{itemize}
\item Analyzing an algorithm has come to mean predicting the resource
  that the algorithm requires.
  \begin{itemize}
  \item Running times of an algorithm on a particular input is the
    number of primitive operations or steps executed and is useful so
    that is is machine or hardware independent as possible.
  \end{itemize}
\item Order of growth of an algorithm gives a simple characteristic of
  the algorithm's efficiency and also allows us to compare the
  relative performance of alternative algorithms.
  \begin{itemize}
  \item Studying the asymptotic efficiency of algorithms is determining
    how running time of an algorithm increases with size of an input in
    the limit as the size of the input increases without bound
  \end{itemize}
\item Big O notation
  \begin{itemize}
  \item $\Omega(f)$ function that grows at least as fast as $f$
  \item $\Theta(f)$ function that grows at the same rate as $f$
  \item $O(f)$ function that grows no faster than $f$
  \end{itemize}
\item Most computer systems represent numbers using strings of binary 
  digits. 
  \begin{itemize}
  \item Numbers often utilize weighted number systems where a set of
    weights are are used to represent a specific position.
    \begin{eqnarray*}
      x  & = & \sum_{i=0}^{n-1} x_i \cdot w_i
    \end{eqnarray*}
    where the weight vector is the following
    \begin{eqnarray*}
      W  & = & (w_{n-1}, \ldots, w_0)
    \end{eqnarray*}
    and the radix (r) is used within the weight.
    \begin{eqnarray*}
      R  & = & (r_{n-1}, \ldots, r_0)
    \end{eqnarray*}
    such that 
    \begin{eqnarray*}
      w_i = r ^ {i}
    \end{eqnarray*}
  \item Weighted number systems are typically cannonical, such that:
    \begin{eqnarray*}
      D_i = {0, 1, 2, \ldots, \mid r_i \mid - 1}
    \end{eqnarray*}
  \item Conventional number systems utilized a fixed radix positive
    and a canonical digit set.
  \end{itemize}
\item In fixed point number systems, the position of the radix point is
  constant. 
\item The two most common fixed point representations are integer
  and fractional. Integer representations are commonly used on 
  general purpose computers, whereas, digital signal processors
  typically use both integer and fractional representations. 
\item  For example, $n$-bit integers have the form:
  \[ A = a_{n-1} a_{n-2} \ldots a_{1} a_{0}. \]
  The most significant bit $a_{n-1}$ is referred to as 
  the {\it sign} bit. If the sign bit is one, the number 
  is negative. Otherwise, it is positive. 
\item Similarly, $n$-bit fractional numbers have the form:
  \[ A = a_{n-1}. a_{n-2} \ldots a_{1} a_{0} \]
\item Fixed Point Binary Integer
  \begin{enumerate}
  \item A $n$-bit conventional fixed-point binary integer has the
    following form:
    \begin{eqnarray*}
      A = \sum_{i=0}^{n-1} a_{i} \cdot 2^{i} 
    \end{eqnarray*}
  \item Most-significant bit (MSB) is the bit to the furthermost left of
    the number.
  \item Least-significant bit (LSB) is the bit to the furthermost right
    of the number.
  \item Unit in the least-signifcant place (ulp) is $r^{-m}$ where $m$
    is the number of digits in the fractional part and $n$ is the number
    of digits in the integer part.
  \end{enumerate}
\item Representation of Negative Numbers
  \begin{itemize}
  \item Sign/Magnitude System
    \begin{itemize}
    \item $a_{n-1}=1$ The number is not positive
    \item $a_{n-1}=0$ The number is not negative
    \end{itemize}
  \item Complement Representation
    \begin{itemize}
    \item Radix Complement (e.g. two's complement)
    \item Diminished-radix Complement (e.g. one's complement)
    \end{itemize}
\end{itemize}
\item 2's complement numbers
  \begin{enumerate}
  \item The value of an $n$-bit 2's complement binary integer is: 
    \[ A = -a_{n-1}2^{n-1} + \sum_{i=0}^{n-2} a_{i} \cdot 2^{i} \]
    For example, with $n = 4$ and $A = 1011.$
    \begin{eqnarray*}
      A  & = & -1 \cdot 2^{3} + 1 \cdot 2^{1} + 1 \cdot 2^{0} \\
      & = & -8 + 2+ 1 \\
      & = & -5
    \end{eqnarray*}
  \item The value of an $n$-bit 2's complement binary fraction is:
    \[ A = -a_{n-1} + \sum_{i=0}^{n-2} a_{i} \cdot 2^{i - n + 1} \]
    For example, with $n = 4$ and $A = 1.011$
    \begin{eqnarray*}
      A  & = & -1 + 0 \cdot 2^{-1} + 1 \cdot 2^{-2} + 1 \cdot 2^{-3} \\
      & = & -1 + 1/4 + 1/8 \\
      & = & -5/8
    \end{eqnarray*}
  \item You can also get the value an $n$-bit binary fraction 
    by computing the value of the $n$-bit binary integer and
    dividing by $2^{n-1}$. 
  \item To negate 2's complement numbers, invert all bits and 
    add a unit in the least position or ulp to the least significant
    bit. For example, 
    \begin{eqnarray*}
      3/8           & = & 0.011 \\
      invert        &   & 1.100 \\
      increment     & + & 0.001 \\
      & = & 1.101 = -3/8 
      % Check:     invert     &   & 0.010 \\
      % 	   increment    & + & 0.001 \\
      %           3/8         &   & 0.011 = 3/8
    \end{eqnarray*}
  \end{enumerate}

\item Sign-magnitude numbers
  \begin{enumerate}
  \item The value of a sign-magnitude binary fraction is:
    \[ A = (1 - 2 \cdot a_{n-1}) \cdot \sum_{i=0}^{n-2} a_{i} \cdot 2^{i - n + 1} \]
    For example, with $n = 4$ and $A = 1.011$
    \begin{eqnarray*}
      A  & = & (1 - 2 \cdot 1) \cdot (0 \cdot  2^{-1} + 1 \cdot 2^{-2} + 1 \cdot 2^{-3}) \\
      & = & -1 \cdot (1/4 + 1/8) \\
      & = & -3/8
    \end{eqnarray*}
  \item To negate sign-magnitude numbers, invert the sign bit (i.e., the most significant bit).
    For example, 
    \begin{eqnarray*}
      3/8           & = & 0.011 \\
      invert\_sign   & = & 1.011 = -3/8 
      % Check:    invert\_sign   & = & 0.011 = 3/8 \\
    \end{eqnarray*}
  \end{enumerate}

\item 1's complement numbers
  \begin{enumerate}
  \item The value of a 1's complement binary fraction is:
    \[ A = \sum_{i=0}^{n-2} (a_{i} - a_{n-1}) \cdot 2^{i - n + 1} \]
    For example, with $n = 4$ and $A = 1.011$
    \begin{eqnarray*}
      A  & = & (0 - 1) \cdot 2^{-1} + (1 - 1) \cdot 2^{-2} + (1 - 1) \cdot 2^{-3}\\
      & = & -1/2 
    \end{eqnarray*}
  \item To negate 1's complement numbers, invert all bits. For example, 
    \begin{eqnarray*}
      3/8           & = & 0.011 \\
      invert        &   & 1.100 = -3/8 \\
      % Check:    invert        &   & 0.011 = 3/8
    \end{eqnarray*}
  \end{enumerate}

\item Characteristics of Fixed Point Number Systems (See Table 1). 
  \begin{enumerate}
  \item Positive numbers are identical for all three number systems. 
  \item The sign-magnitude and 1's complement number systems have both $+0$ and $-0$. 
  \item Only the 2's complement number system has a representation for $-1$. 
  \end{enumerate}

\item Uses of Fixed Point Number Systems
  \begin{enumerate}
  \item The 2's complement number system is the most common format 
    for fixed-point numbers (all operations are fairly easy)
  \item The sign-magnitude number system is the most common format for 
    floating-point numbers (lets sign be handled separately and
    provides both +0 and -0)
  \item The 1's complement number system is not frequently used
  \end{enumerate}

\item Behavior under truncation
  \begin{enumerate}
  \item 2's complement number truncate toward $ -\infty $. The number either
    decreases in value or stays the same. For example,  
    \begin{eqnarray*}
      0.011 & = & 3/8 \\ 
      0.01  & = & 2/8 = 1/4 \\ 
      1.011 & = & -5/8 \\ 
      1.01  & = & -6/8 = -3/4
    \end{eqnarray*}
  \item Sign-magnitude numbers truncate toward 0. Positive numbers either
    decrease in value or stays the same, but negative numbers either
    increase in value or stay the same. For example,  
    \begin{eqnarray*}
      0.011 & = & 3/8 \\ 
      0.01  & = & 2/8 = 1/4 \\ 
      1.011 & = & -3/8 \\ 
      1.01  & = & -2/8 = -1/4
    \end{eqnarray*}
  \item 1's complement numbers also truncate toward 0. For example,  
    \begin{eqnarray*}
      0.011 & = & 3/8 \\ 
      0.01  & = & 2/8 = 1/4 \\ 
      1.011 & = & -1/2 \\ 
      1.01  & = & -1/2 = -4/8 \\ 
      1.010 & = & -5/8 \\ 
      1.01  & = & -1/2 = -4/8 \\ \\
    \end{eqnarray*}    
  \end{enumerate}
\end{enumerate}

\begin{table*} [t!]
  \centering
  \begin{tabular}{|c|c|c|c|} \hline
    Number  & 2's Complement & Sign Magnitude & 1's Complement \\ \hline \hline
    +7/8   &      0.111     &    0.111       &      0.111      \\ \hline
    +3/4   &      0.110     &    0.110       &      0.110      \\ \hline
    +5/8   &      0.101     &    0.101       &      0.101      \\ \hline
    +1/2   &      0.100     &    0.100       &      0.100      \\ \hline
    +3/8   &      0.011     &    0.011       &      0.011      \\ \hline
    +1/4   &      0.010     &    0.010       &      0.010      \\ \hline
    +1/8   &      0.001     &    0.001       &      0.001      \\ \hline
    +0     &      0.000     &    0.000       &      0.000      \\ \hline
    -0     &       N/A      &    1.000       &      1.111      \\ \hline
    -1/8   &      1.111     &    1.001       &      1.110      \\ \hline
    -1/4   &      1.110     &    1.010       &      1.101      \\ \hline
    -3/8   &      1.101     &    1.011       &      1.100      \\ \hline
    -1/2   &      1.100     &    1.100       &      1.011      \\ \hline
    -5/8   &      1.011     &    1.101       &      1.010      \\ \hline
    -3/4   &      1.010     &    1.110       &      1.001      \\ \hline
    -7/8   &      1.001     &    1.111       &      1.000      \\ \hline
    -1     &      1.000     &     N/A        &       N/A       \\ \hline
  \end{tabular}
  \caption{$4$-bit Fixed Point Fractions}
\end{table*}

\end{document}
