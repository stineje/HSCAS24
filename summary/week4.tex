\documentclass[times, twocolumn, 10pt]{article}
\usepackage{url}
\usepackage{fancyhdr}
\usepackage{amsmath, amsfonts, amsthm, amssymb}  
\usepackage{epsfig}
\usepackage{lastpage}
\usepackage{hyperref}
\usepackage{graphicx}
\usepackage{color}
\usepackage{epstopdf}
\usepackage{fancyvrb}

\usepackage{geometry}
\geometry{letterpaper, left=1in, right=1in, top=1in, bottom=1in}

\pagestyle{fancy}
\fancyhf{}
\renewcommand{\headrulewidth}{0pt}
\rfoot{\thepage/\pageref{LastPage}}
%\lhead{OSU ECEN 4233 - High-Speed Computer Arithmetic - Spring 2021}
\lfoot{\LaTeX}

\begin{document}

\title{Dual Half/Full, Carry-Increment, and Conditional Sum Adders}
\author{James E. Stine \\
Electrical and Computer Engineering Department\\
Oklahoma State University \\
Stillwater, OK 74078, USA}
\date{}

\maketitle
%\thispagestyle{plain}\pagestyle{plain}


\begin{enumerate}
\item Dual Half Adders
  \begin{enumerate}
  \item A Dual Half Adder (DHA) generates the sum and carry bits for
    each position. 
  \item The logic equations for a DHA are 
    \begin{eqnarray*}
      s_{k}^{0}   & = & a_{k} \oplus b_{k} \\
      s_{k}^{1}   & = & \overline{a_{k} \oplus b_{k}} = \overline{s_{k}^{0}} \\
      c_{k+1}^{0} & = & a_{k} \cdot b_{k} \\
      c_{k+1}^{1} & = & a_{k} + b_{k}
    \end{eqnarray*}
  \item The delays for a DHA are  
    \begin{eqnarray*}
      a_{k}, b_{k} \rightarrow s_{k}^{0} & = & 3 \bigtriangleup \\
      a_{k}, b_{k} \rightarrow s_{k}^{1} & = & 4 \bigtriangleup \\
      a_{k}, b_{k} \rightarrow c_{k+1}^{0}, c_{k+1}^{1} & = & 1 \bigtriangleup 
    \end{eqnarray*}
  \end{enumerate}
  \begin{figure*} [h!]
    \begin{center}
      \setlength{\unitlength}{0.0105in}%
      \epsfig{figure=dha.eps, height=1.0in}
    \end{center}
    \label{dha.fig}
    \caption{Dual Half Adder.}
  \end{figure*}
\item Dual Ripple Carry Adders (DRCA) 
  \begin{enumerate}
  \item A DRCA computes one set of sum and carry bits with a carry in 
    of zero, and a second set with a carry in of one. 

  \item A single $r$-bit DRCA can replace two $r$-bit RCAs, when 
    constructing a Carry Select Adder

  \item The first bit of the DRCA uses a Dual Half Adder (DHA). 

  \item A DHA uses 5 gates and has the following delays:
    \begin{eqnarray*}
      a_{k}, b_{k} \rightarrow s_{k}^{0} & = & 3 \bigtriangleup \\
      a_{k}, b_{k} \rightarrow s_{k}^{1} & = & 4 \bigtriangleup \\
      a_{k}, b_{k} \rightarrow c_{k+1}^{0}, c_{k+1}^{1} & = & 1
      \bigtriangleup
    \end{eqnarray*}

  \item All of the other bits use dual full adders (DFA). 

  \item A DFA uses $2 \cdot 5 + 4 = 14$ gates and has the following
    gate delays: 
    \begin{eqnarray*}
      a_{k}, b_{k} \rightarrow s_{k}^{0} = 7 \bigtriangleup \\
      a_{k}, b_{k} \rightarrow s_{k}^{1} = 6 \bigtriangleup \\
      a_{k}, b_{k} \rightarrow c_{k+1}^{0} = 5 \bigtriangleup \\
      a_{k}, b_{k} \rightarrow c_{k+1}^{1} = 6 \bigtriangleup \\
      c_{k}^{0} \rightarrow s_{k}^{0} = 3 \bigtriangleup \\
      c_{k}^{1} \rightarrow s_{k}^{1} = 3 \bigtriangleup \\
      c_{k}^{0} \rightarrow c_{k+1}^{0} = 2 \bigtriangleup \\
      c_{k}^{1} \rightarrow c_{k+1}^{1} = 2 \bigtriangleup 
    \end{eqnarray*}

  \item An $n$-bit DRCA requires $1$ DHA and $(n-1)$ DFAs.  That is,
    it requires $5 + (n-1) \cdot 14$ or $14 \cdot n - 9$ gates and has
    the following delays. 
    \begin{eqnarray*}
      a_{k}, b_{k} \rightarrow s_{n-1}^{0}, s_{n-1}^{1} = (2n+3)
      \bigtriangleup \\
      a_{k}, b_{k} \rightarrow c_{n}^{0}, c_{n}^{1} = (2n+2) \bigtriangleup 
    \end{eqnarray*}
  \end{enumerate}
  \begin{figure*} [h!]
    \begin{center}
      \setlength{\unitlength}{0.0105in}%
      \epsfig{figure=dfa.eps, height=2.0in}
    \end{center}
    \label{dfa.fig}
    \caption{Dual Full Adder.}
  \end{figure*} [h!]
  \begin{figure*}
    \begin{center}
      \setlength{\unitlength}{0.0105in}%
      \epsfig{figure=drca.eps, height=1.5in}
    \end{center}
    \label{drca.fig}
    \caption{Dual Ripple Carry Adder.}
  \end{figure*}


\item Carry Select Adders with DRCA 
  \begin{enumerate}
  \item The DRCA can obviously be integrated with the Carry Select Adder to
    reduce the overall area content.  
  \end{enumerate}
  \begin{figure*} [p]
    \begin{center}
      \setlength{\unitlength}{0.0105in}%
      \epsfig{figure=sel16a.eps, height=1.5in}
    \end{center}
    \label{sel16a.fig}
    \caption{$16$-bit Carry Select Adder using DRCA.}
  \end{figure*}

\item Carry Increment Adder (CINA)
\begin{enumerate}
\item The carry increment adder was refined by R. Zimmermann and
  basically formulated by A. Takagi titled, ``A reduced-area scheme for
  carry-select adders'' in 1993.
\item The idea is an algorithmic enhancement of carry-select adders using the
  ideas from the Weinberger-Smith Carry-Lookahead Concept.
\item The basic proof of enhanced equation requires adding a redundant
  term to the equation we formulated with the DHA.  That is, 
  it first requires using the following equation, which we will
  attempt to prove:
  \begin{eqnarray*}
    c_{k+1}^1 & = & c_{k+1}^0 + c_{k+1}^1
  \end{eqnarray*}
  To prove this equation, we break part the equations from the DHA
  definitions.  
  \begin{eqnarray*}
    c_{k+1}^{1} & = & (g_k + p_k \cdot c_k^{1}) + (g_k + p_k \cdot
    c_k^{0})  \\
    c_{k+1}^{1} & = & g_k + g_k + p_k \cdot {c_k}^{1} + p_k \cdot c_k^{0} \\
    c_{k+1}^{1} & = & g_k + p_k \cdot c_k^{1} + p_k \cdot c_k^{0} \\
    c_{k+1}^{1} & = & g_k + p_k \cdot (a_{k-1} + b_{k-1} + 
    a_{k-1} \cdot b_{k-1}) \\
    c_{k+1}^{1} & = & g_k + p_k \cdot (a_{k-1} \cdot b_{k-1}) =
    c_{k+1}^1
  \end{eqnarray*}

\item This proof basically uses two elements of DHA, such that:
  \begin{eqnarray*}
    c_{k+1}^0 & = & a_k \cdot b_k \\
    c_{k+1}^1 & = & a_k + b_k 
  \end{eqnarray*}

\item So, the CINA uses the above formulation in the idea that carry-out or
  $c_{k+1}$ can be selected between two values inside a multiplexor or mux.
  That is, the following relationship holds:
\begin{eqnarray*}
c_{k+1} & = & \overline{c_{in}} \cdot c_{k+1}^0 + c_{in} \cdot c_{k+1}^1 
\end{eqnarray*}

\item Using this carry-out equation and the new redundant-enhanced equation
  forms the following relationship:
\begin{eqnarray*}
c_{k+1} & = & \overline{c_{in}} \cdot c_{k+1}^0 + c_{in} \cdot (c_{k+1}^0 +
    c_{k+1}^1) \\
c_{k+1} & = & c_{k+1}^0 + c_{in} \cdot c_{k+1}^1 \\
c_{k+1} & = & (g_k + p_k \cdot {c_k}^0) + c_{in} \cdot (g_k + p_k \cdot c_k^1)
    \\
c_{k+1} & = & c_{k+1}^0 + c_{in} \cdot p_k \cdot c_k^1 \\
c_{k+1} & = & c_{k+1}^0 + c_{in} \cdot p_{k:k-r}
\end{eqnarray*}
\item The previous simplification is achieved due to the
  convering theorem in Boolean logic, such that:  $g_k + c_{in} \cdot g_k = g_k$.
\item The last item assumes the basic ideas of the carry-select adder in that
  a carry-in of 1 (i.e., $c_k^1$) produces the group-propagate signal.
  Therefore, the RCA that produces the correct sum is incremented based on
  the value of $c_{k+1}$.
\item Generalized CINA Gate Counts
\begin{enumerate}
\item The first block, similar the carry-select adder, is nothing more than a
  carry-propagate ader.
\item Each subsequent $r$ block uses a RCA to produce the correct sum as well
  as the bitwise propagate signals from a $9$ gate FA.
\item It uses $\lceil n/r \rceil - 1$ sets of carry logic blocks, each of
  which carries $2$ gates.
\item The second RCA is a RCA that is composed of HA blocks adding $s_k^0$
  and $c_{k+1}$ from the previous block.  Since the RCA just needs a $c_{in}$,
  the number of gates is $4 \cdot (n - r)$.
\item Thus, the total number of gates used by a $n$-bit CINA with $r$-bit
  blocks is:
\begin{eqnarray*}
  & = & 9 \cdot n + 2 \cdot \left( \lceil \frac{n}{r} \rceil -1
  \right) + 4 \cdot (n - r) \\
  & = & 9 \cdot n + 2 \cdot \lceil \frac{n}{r} \rceil - 2 + 4 \cdot n
  - 4 \cdot r \\
  & = & 13 \cdot n + 2 \cdot \lceil \frac{n}{r} \rceil - 2 - 4 \cdot r
\end{eqnarray*}
\end{enumerate}
\item Generalized CINA Delay
\begin{enumerate}
\item The first block is a RCA and has a delay of $2 \cdot r + 3$
  \item The next $(\lceil n/r \rceil - 1)$ blocks have a delay of 
    $2 \bigtriangleup$  for the carry to skip. 
\item The last block must go through the RCA incrementor, which is composed
  of HAs aligned like a RCA.  The delay of this structure is
\begin{eqnarray*}
(r-1) \cdot 1 + 3 = (r + 2)~\bigtriangleup
\end{eqnarray*}
\item Thus, the total delay for $s_{n-1}$ is:
  \begin{eqnarray*}
    & = & (2 \cdot r + 3) + 2 \cdot (\lceil \frac{n}{r} \rceil - 2) + (r + 2)
    \\
    & = & 3 \cdot r + 2 \cdot \lceil \frac{n}{r} \rceil + 1
    \\
  \end{eqnarray*}
\end{enumerate}
\end{enumerate}
\begin{figure*} [p]
  \begin{center}
    \setlength{\unitlength}{0.0105in}%
    \epsfig{figure=cina16.eps, height=1.5in}
  \end{center}
  \label{cina16.fig}
  \caption{$16$-bit Carry Increment Adder with $r=4$.}
\end{figure*}

\item Conditional Sum Concept. 
  \begin{enumerate}
  \item The conditional sum adder (CSUA) generates a pair of sum and carry bits
    is generated at each bit position. One pair assumes carry in of
    one and the other assumes a carry in of zero. 
  \item The correct sums and carries are then selected using a tree of
    multiplexors. 
  \end{enumerate}
\item Example of Conditional Sum Addition
  \begin{table} [h]
    \centering
    \label{csua.tbl}
    \begin{tabular}{|c|c|c|c|c|c|c|c|c|c|} \hline
      stage & $k $          & 7 & 6 & 5 & 4 & 3 & 2 & 1 & 0  \\ \hline 
      0     & $a_{k}$       & 1 & 0 & 1 & 1 & 0 & 1 & 1 & 0  \\ 
      & $b_{k}$       & 0 & 0 & 1 & 0 & 1 & 1 & 0 & 1  \\ \hline
      1     & $s_{k}^{0}$   & 1 & 0 & 0 & 1 & 1 & 0 & 1 & 1  \\
      & $c_{k+1}^{0}$ & 0 & 0 & 1 & 0 & 0 & 1 & 0 & 0  \\ \hline
      & $s_{k}^{1}$   & 0 & 1 & 1 & 0 & 0 & 1 & 0 &    \\
      & $c_{k+1}^{1}$ & 1 & 0 & 1 & 1 & 1 & 1 & 1 &    \\ \hline
      2     & $s_{k}^{0}$   & 1 & 0 & 0 & 1 & 0 & 0 & 1 & 1  \\
      & $c_{k+1}^{0}$ & 0 &   & 1 &   & 1 &   & 0 &    \\ \hline
      & $s_{k}^{1}$   & 1 & 1 & 1 & 0 & 0 & 1 &   &    \\
      & $c_{k+1}^{1}$ & 0 &   & 1 &   & 1 &   &   &    \\ \hline
      3     & $s_{k}^{0}$   & 1 & 1 & 0 & 1 & 0 & 0 & 1 & 1  \\
      & $c_{k+1}^{0}$ & 0 &   &   &   & 1 &   &   &    \\ \hline
      & $s_{k}^{1}$   & 1 & 1 & 1 & 0 &   &   &   &    \\
      & $c_{k+1}^{1}$ & 0 &   &   &   &   &   &   &    \\ \hline
      4     & $s_{k}    $   & 1 & 1 & 1 & 0 & 0 & 0 & 1 & 1  \\
      & $c_{k+1}    $ & 0 &   &   &   &   &   &   &    \\ \hline
    \end{tabular}
  \end{table}
\item An $8$-bit Conditional Sum Adder 
  \begin{enumerate}
  \item An $8$-bit CSUA requires one row of DHAs and three levels of multiplexors. 
  \item The first row has a delay of $5 \bigtriangleup$ to produce the 
    carry out of the full adder, and the next three rows each have 
    a delay of $4 \bigtriangleup$ for the muxes. Thus, the total
    delay for an $8$-bit CSUA is $5 + 3 \cdot 4 = 17 \bigtriangleup$ 
  \item The first row requires $7$ DHAs ($5$ gates each) and $1$ FA
    ($9$ gates). 
    The second row requires $7$ mux21x2 ($8$ gates each). 
    The third row requires $3$ mux21x3 ($12$ gates). 
    The fourth row requires $1$ mux21x5 ($20$ gates). 
    Thus, the total number of gates is 
    \[ 	7 \cdot 5 + 9 + 7 \cdot 8 + 3 \cdot 12 + 20 = 156 \]
  \end{enumerate}
\item Generalized  CSUA Gate Count (assume $n$ is a power of 2)
  \begin{enumerate}
  \item An $n$-bit CSUA uses $n - 1$ DHAs and 1 FA in the first level. 
  \item After this, there are $\lceil \log_{2}(n) \rceil$ rows of muxes. 
    If the mux rows are labeled with indices 
    $i = 0, 1, \ldots,  \lceil log_{2}(n) \rceil - 1$,
    then in row $i$ number of muxes is ($\frac{n}{2^{i}} - 1$), the number of
    bits per mux is $2^{i} + 1$, and the number of gates per mux is 
    $4(2^{i} + 1) = 4 \cdot 2^{i} + 4$.
  \item Thus the toal number of gates for an $n$-bit CSUA is 
    \begin{eqnarray*}
      & & 5 \cdot (n - 1) + 9 +\\
      &  & \sum_{i=0}^{\lceil log_{2}(n) \rceil -1}
      (\frac{n}{2^{i}} - 1) (4 \cdot 2^{i} + 4) \\
      & \approx & 4 \cdot n \cdot \log_{2}(n) + 9 \cdot n - 4 \cdot
      log_{2}(n) 
    \end{eqnarray*}
  \end{enumerate}
\item Generalized CSUA Delay
  \begin{enumerate}
  \item The first row has a delay of $5~\bigtriangleup$ to produce the 
    carry out of the full adder. 
  \item  Each mux row has a delay of $3~\bigtriangleup$ and there 
    are $\lceil \log_{2}(n) \rceil$ rows of muxes. 
  \item  Thus, the total delay is $(5 + 3 \cdot \lceil \log_{2}(n)
    \rceil)~\bigtriangleup$.
  \end{enumerate}
\item Characteristics of CSUAs
  \begin{enumerate}
  \item CSUAs require the largest number of gates. The have
    $O(n \cdot log_{2}(n))$ area, whereas, the other adders (beside the CLA)
    had $O(n)$ area. 
  \item Like the CLA, CSUAs have logarithmic delay. However, there
    delay is always base two, whereas, the delay for the CLA 
    is base $r$, where $r$ is the maximum number of inputs. 
  \item For $r = 2$ CSUAs become faster than CLAs, as $n$
    gets large. 
  \item CSUAs are highly irregular and difficult to layout. Consequently, 
    they are seldom implemented in practice. 
  \end{enumerate}

\item Hybrid Carry Select Adders (HCSEA)
  \begin{enumerate}
  \item With a HCSEA, the first adder is a carry lookahead adder, 
    and a carry lookahead generator is used to produce the carries. 
  \item 16-bit HCSEA with $4$ bit blocks has a worst case delay of $16
    \bigtriangleup$ and uses $310$ gates. 
  \item If DRCAs are used, the worst case delay is reduced to $15
    \bigtriangleup$ and only $254$ gates are required. 
  \end{enumerate}
  \begin{figure*} [p]
    \begin{center}
      \setlength{\unitlength}{0.0105in}%
      \epsfig{figure=hybrid_sel16.eps, height=3.5in}
    \end{center}
    \label{hybrid_sel16.fig}
    \caption{16-bit Hybrid Carry Select Adder.}
  \end{figure*}
  
\item{Power Dissipation in Adders}
  \begin{enumerate}
  \item The main source of power dissipation in well designed CMOS circuits
    is due to low-to-high logic transitions in digital circuits. 
  \item This power dissipation, sometimes called dynamic power
    dissipation, can be expressed as
    \[	P = V_{DD}^{2} \cdot f_{clk} \cdot C_{eff} \cdot p_{t} \]
    where $V_{DD}$ is the source voltage, $f_{clk}$ is the clock frequency, 
    $C_{eff}$ is the effective capacitance and $p_{t}$ is the probability
    of a low-to-high logic transition. 
  \item Dynamic power dissipation can be
    reduced by reducing any of the above $4$ factors more or less. 
  \item If all other factors are constant, then the probability of low-to-high
    logic transitions is a good measure of the total power.     
  \item The actual power consumption ranks and the output transitions ranks are 
    fairly close. The main exceptions to this is the Carry Lookahead Adder and
    the Majerski Ripple Adder. Both of these adders have gates with 
    higher fan-in, which increases the capacitance.
  \item There is still some significant research that needs to be done in this area.    
  \end{enumerate}
\end{enumerate}

\begin{center}
\begin{table*}[ht]
  \centering
  \caption{$16$-Bit Adder Area~\cite{378090}.}
  \label{par_add_area}
  \begin{tabular}{||l|r|r|r|r||} \hline
    Adder Type              & Area ($\mbox{mm}^{2}$) & rank & Gate Count & rank \\ \hline
    Ripple Carry            & 0.26 & 2 & 144 & 2 \\ \hline
    Majerski Ripple Carry   & 0.23 & 1 & 122 & 1 \\ \hline
    Constant Carry Skip     & 0.33 & 3 & 156 & 3 \\ \hline
    Variable Carry Skip     & 0.49 & 4 & 170 & 4 \\ \hline
    Carry Lookahead         & 0.53 & 5 & 200 & 5 \\ \hline
    Brent and Kung          & 0.53 & 6 & 203 & 6 \\ \hline
    Hybrid Carry Select     & 0.88 & 7 & 284 & 7 \\ \hline
    Conditional Sum         & 1.14 & 8 & 368 & 8 \\ \hline
  \end{tabular}
\end{table*}
\begin{table*}[ht]
  \centering
  \caption{$16$-Bit Adder Delay~\cite{378090}.}
  \label{par_add_del_16_simp}
  \begin{tabular}{||l|r|r|r|r||} \hline
    Adder Type              & Delay (nsec) & rank & Gate Delay & rank \\ \hline
    Ripple Carry            & 51.4 & 8 & 36 & 8 \\ \hline
    Majerski Ripple Carry   & 34.3 & 7 & 19 & 6 \\ \hline
    Constant Carry Skip     & 28.6 & 6 & 23 & 7 \\ \hline
    Variable Carry Skip     & 22.8 & 5 & 17 & 4 \\ \hline
    Carry Lookahead         & 22.5 & 4 & 10 & 1 \\ \hline
    Brent and Kung          & 22.1 & 3 & 18 & 5 \\ \hline
    Hybrid Carry Select     & 18.6 & 1 & 14 & 3 \\ \hline
    Conditional Sum         & 21.2 & 2 & 12 & 2 \\ \hline
  \end{tabular}
\end{table*}
\begin{table*}[ht]
  \centering
  \caption{$16$-Bit Adder Power~\cite{378090}.}
  \label{par_add_pow_16_simp}
  \begin{tabular}{||l|r|r|r|r||} \hline
    Adder Type              & Power   & rank & Gate Output & rank \\ 
    & (mWatt) &      & Transitions & \\ \hline
    Ripple Carry            & 1.7 & 1 & 90  & 1 \\ \hline
    Majerski Ripple Carry   & 2.7 & 4 & 91  & 2 \\ \hline
    Constant Carry Skip     & 1.8 & 2 & 99  & 3 \\ \hline
    Variable Carry Skip     & 2.2 & 3 & 108 & 5 \\ \hline
    Carry Lookahead         & 2.7 & 5 & 100 & 4 \\ \hline
    Brent and Kung          & 3.1 & 6 & 112 & 6 \\ \hline
    Hybrid Carry Select            & 3.8 & 7 & 150 & 7 \\ \hline
    Conditional Sum         & 5.4 & 8 & 231 & 8 \\ \hline
  \end{tabular}
\end{table*}
\end{center}
\begin{figure*} [p]
  \begin{center}
    \setlength{\unitlength}{0.0105in}%
    \epsfig{figure=cond8.eps, height=5.0in, angle=-90}
  \end{center}
  \label{cond8.fig}
  \caption{8-bit Conditional Sum Adder.}
\end{figure*}

\bibliographystyle{ieeetr}
\bibliography{week4}


\end{document}
