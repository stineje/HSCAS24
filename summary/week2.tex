\documentclass[times, twocolumn, 10pt]{article}
\usepackage{url}
\usepackage{fancyhdr}
\usepackage{amsmath, amsfonts, amsthm, amssymb}  
\usepackage{epsfig}
\usepackage{lastpage}
\usepackage{hyperref}
\usepackage{graphicx}
\usepackage{color}
\usepackage{epstopdf}
\usepackage{fancyvrb}

\usepackage{geometry}
\geometry{letterpaper, left=1in, right=1in, top=1in, bottom=1in}

\pagestyle{fancy}
\fancyhf{}
\renewcommand{\headrulewidth}{0pt}
\rfoot{\thepage/\pageref{LastPage}}
%\lhead{OSU ECEN 4233 - High-Speed Computer Arithmetic - Spring 2021}
\lfoot{\LaTeX}

\begin{document}

\title{Basic Adder Structures and the Lookahead Concept}
\author{James E. Stine \\
  Electrical and Computer Engineering Department\\
  Oklahoma State University \\
  Stillwater, OK 74078, USA}
\date{}

\maketitle
%\thispagestyle{plain}\pagestyle{plain}
\begin{enumerate}
\item Binary Addition and Subtraction Operations
  \begin{enumerate}
  \item In fixed point number systems, the simplest arithmetic operations 
    are addition and subtraction. 
  \item To compute the sum of two 2's complement numbers ($S = A + B$),
    add the corresponding bits of both numbers. For example, 
    \begin{eqnarray*}
      A  & = & 0.1011 = 11/16 \\
      + B  & = & 0.0011 =  3/16 \\
      = S  & = & 0.1110 = 14/16 
    \end{eqnarray*}
    and 
    \begin{eqnarray*}
      A  & = & 0.1001 =  9/16 \\
      + B  & = & 1.0100 =  -12/16 \\
      = S  & = & 1.1101 = -3/16 
    \end{eqnarray*}
  \item To compute the difference of two 2's complement numbers ($S = A - B$),
    add the bits of the minuend $A$ to the 2's complement of the subtrahend
    $B$. For example, 
    \begin{eqnarray*}
      A  & = & 0.1011 = 11/16 \\
      - B  & = & 0.0011 =  3/16 
    \end{eqnarray*}
    becomes 
    \begin{eqnarray*}
      A  & = & 0.1011 =  11/16 \\
      +(-B) & = & 1.1101 =  -3/16 \\
      = S  & = & 0.1000 = 8/16
    \end{eqnarray*}
  \end{enumerate}
\item Half Adders 
  \begin{enumerate}
  \item A fundamental building block in arithmetic systems is the half
    adder (HA). 
  \item A HA takes two bits $a_{k}$ and $b_{k}$ and produces a sum bit
    $s_{k}$ and
    a carry bit $c_{k+1}$. 
    \begin{table} [h]
      \centering
      \label{ha.tbl}
      \begin{tabular}{|c|c||c|c|} \hline
	$a_{k}$ & $b_{k}$ & $c_{k+1}$ & $s_{k}$  \\ \hline \hline
        0    &    0    &    0       &   0      \\ \hline
        0    &    1    &    0       &   1      \\ \hline
        1    &    0    &    0       &   1      \\ \hline
        1    &    1    &    1       &   0      \\ \hline
      \end{tabular}
      \caption{HA Truth Table}
    \end{table}
  \item The logic equations for a HA are:
    \begin{eqnarray*}
      s_{k}  & = & \overline{a_{k}} \cdot b_{k} + a_{k} \cdot \overline{b_{k}} \\
      & = & a_{k} \oplus b_{k} \\
      c_{k+1}  & = &  a_{k} \cdot b_{k} 
    \end{eqnarray*}
  \item One possible logic implementation of the HA is shown in
    Figure~\ref{ha.fig}
    This HA requires 4 logic gates and has the following delays: 
    \begin{eqnarray*}
      a_{k}, b_{k} \rightarrow s_{k} & = & 3 \bigtriangleup \\
      a_{k}, b_{k} \rightarrow c_{k+1} & = & 1 \bigtriangleup
    \end{eqnarray*}
    where $\bigtriangleup$ represents a single gate delay.  
    \begin{figure*}
      \label{ha.fig}
      \begin{center}
	\setlength{\unitlength}{0.0105in}%
	\epsfig{figure=ha.eps,height=1.0in}
      \end{center}
      \caption{HA Implementation.}
    \end{figure*}
  \item Full Adders
    \begin{enumerate}
    \item A second building block in arithmetic systems is the full
      adder (FA). 
    \item A FA takes three bits $a_{k}$, $b_{k}$, and $c_{k}$ and
      produces a sum 
      bit $s_{k}$ and a carry bit $c_{k+1}$. 
    \item The truth table for a FA is:
      \begin{table} [h]
	\centering
	\label{fa.tbl}
	\begin{tabular}{|c|c|c||c|c|} \hline
	  $a_{k}$ & $b_{k}$ & $c_{k}$ & $c_{k+1}$ & $s_{k}$  \\ \hline \hline
          0    &    0     &   0     &   0       &   0      \\ \hline
          0    &    0     &   1     &   0       &   1      \\ \hline
          0    &    1     &   0     &   0       &   1      \\ \hline
          0    &    1     &   1     &   1       &   0      \\ \hline
          1    &    0     &   0     &   0       &   1      \\ \hline
          1    &    0     &   1     &   1       &   0      \\ \hline
          1    &    1     &   0     &   1       &   0      \\ \hline
          1    &    1     &   1     &   1       &   1      \\ \hline
	\end{tabular}
	\caption{FA Truth Table}
      \end{table}
    \item The logic equations for a FA are:
      \begin{eqnarray*}
	s_{k}  & = & \overline{a_{k}} \cdot \overline{b_{k}} \cdot c_{k} + \
	\overline{a_{k}} \cdot b_{k} \cdot \overline{c_{k}} + \\
	& & a_{k} \cdot \overline{b_{k}} \cdot \overline{c_{k}} + \
	a_{k} \cdot b_{k} \cdot c_{k} \\
	& = & a_{k} \oplus b_{k} \oplus c_{k} \\
	c_{k+1}  & = &  \overline{a_{k}} \cdot b_{k} \cdot c_{k} + \
	a_{k} \cdot \overline{b_{k}} \cdot c_{k} + \\
	& & a_{k} \cdot b_{k} \cdot \overline{c_{k}} + \
	a_{k} \cdot b_{k} \cdot c_{k} \\
	& = &  a_{k} \cdot b_{k} +  a_{k} \cdot c_{k} +  b_{k} \cdot c_{k} 
      \end{eqnarray*}
    \item A FA can be constructed from two HAs and one OR gate, as
      shown in Figure~\ref{fa.fig}.
      This FA requires 9 logic gates and has the following delays: 
      \begin{eqnarray*}
	a_{k}, b_{k} \rightarrow s_{k} = 6 \bigtriangleup \\
	a_{k}, b_{k} \rightarrow c_{k+1} = 5 \bigtriangleup \\
	c_{k} \rightarrow s_{k} = 3 \bigtriangleup \\
	c_{k} \rightarrow c_{k+1} = 2 \bigtriangleup \\
      \end{eqnarray*}
      \begin{figure*}
	\label{fa.fig}
	\begin{center}
	  \setlength{\unitlength}{0.0105in}%
	  \epsfig{figure=fa.eps, height=1.25in}
	\end{center}
	\caption{FA Implementation.}
      \end{figure*}
    \item Ripple Carry Adders
      \begin{enumerate}
      \item Ripple carry adders (RCA) provides a slow, but hardware
        efficient, method for adding 
	two binary numbers. 
      \item An $n$-bit RCA is formed by concatenating $n$ FAs. 
      \item The carry out from the $k^{th}$ FA is used as the carry in of 
	the $(k + 1)^{th}$ FA, as shown in Figure~\ref{rca.fig}. 
	\begin{figure*}
	  \label{rca.fig}
	  \begin{center}
	    \setlength{\unitlength}{0.0105in}%
	    \epsfig{figure=rca.eps, height=1.25in}
	  \end{center}
	  \caption{RCA Implementation.}
	\end{figure*}
      \item Since an $n$-bit RCA requires $n$ FAs and each FA has nine gates, 
	the total number of gates for the $n$-bit RCA is $9n$. The worst case 
	delays for a ripple carry adder are:
	\begin{eqnarray*}
	  a_{0}, b_{0} \rightarrow s_{n-1} = (2n + 4) \bigtriangleup \\
	  a_{0}, b_{0} \rightarrow c_{n} = (2n + 3) \bigtriangleup
	\end{eqnarray*}
      \end{enumerate}
    \item Ripple Carry Adders/Subtracters
      \begin{enumerate}
      \item To perform subtraction, it is necessary to complement each
        of the bits
	of $B$ and add a one to the least significant bit. This is accomplished
	by adding a row of $n$ XOR gates to form a ripple-carry
        adder/subtractor (RCSA), 
	as shown in Figure~\ref{rcas.fig}. 
	\begin{figure*} [t!]
	  \label{rcas.fig}
	  \begin{center}
	    \setlength{\unitlength}{0.0105in}%
	    \epsfig{figure=rcas.eps, height=1.75in}
	  \end{center}
	  \caption{RCAS Implementation.}
	\end{figure*}
      \item To perform subtraction, {\it sub} is set to one, which
        complements the bits 
	of $B$ and sets $c_{0}$ to one. To perform addition, {\it sub}
        is set to 
	zero, which leaves the bits of $B$ unchanged and sets $c_{0}$ to zero. 
      \item If an XOR gate is equivalent to 4 simple gates, and has a
        delay of $3 \bigtriangleup$,
	the RSCA requires $13n$ gates and has the following worst case delays: 
	\begin{eqnarray*}
	  b_{0} \rightarrow s_{n-1} & = & (2n + 7) \bigtriangleup \\
	  b_{0} \rightarrow c_{n} & = & (2n + 6) \bigtriangleup
	\end{eqnarray*}
      \end{enumerate}
    \item Overflow Detection
      \begin{enumerate}
      \item Overflow occurs when you compute an operation with two valid
        representations and the result cannot be represented for the
        given
        representation because the value is too large or too small.
      \item Specific detection of overflow requires knowing which
        operation is being performed and its given representation
        (e.g., two's complement).
        \item For an unsigned number representation, a simple way
          of detecting overflow is add two numbers and check whether
          the result is larger than the largest posible value (e.g.,
          for an $n$-bit integer this would be $2^n-1$).  This is
          because it would take $n+1$ bits to represent the result
          correctly.  
          \item If $A + B$ or $ A - B$ is greater than or equal to +1, or
            less than $-1$,
	    overflow occurs and the computed sum or difference is
            incorrect. For example, 
	    \begin{eqnarray*}
	      A  & = & 0.1001 =  9/16 \\
	      + B  & = & 0.1100 =  12/16 \\
	      = S  & = & 1.0101 =  -13/16 
	    \end{eqnarray*}
	    is an incorrect result
          \item One way to detect overflow is to check the result
            of the sum. However,
            this can be complicated because as the input operand grows,
            it requires the result before it can determine an
            overflow, which is computationally complex.
          \item A simple method to detect overflow by comparing the
            the carry into the
            most significant bit
	    to the carry out of the most significant bit. If these two
            bits differ, 
	    then overflow has occurred. This can be expressed as:
	    \[  OV = c_{n-1} \oplus c_{n} \] 
	    For example, in the previous computation of 
            $A =  9/16$ added to $B = 12/16$:
	    \begin{eqnarray*}
	      c_{n-1} & = & 1 \\
	      c_{n}   & = & 0 \\
	      OV      & = & 1 \oplus 0 = 1 
	    \end{eqnarray*}
	    and overflow is detected. 
      \end{enumerate}
    \end{enumerate}
  \end{enumerate}
  \begin{enumerate}
    % CLA Concept
  \item Carry Lookahead Concept. 
    \begin{enumerate}
    \item A carry lookahead adder uses specialized logic to compute 
      the carries in parallel. 
    \item A full adder generates a carry if both $a_{k}$ and $b_{k}$
      are one. 
      \begin{eqnarray*}
	g_{k} = a_{k} \cdot b_{k}      
      \end{eqnarray*}
    \item A full adder propagates a carry if either $a_{k}$ or
      $b_{k}$ is one. 
      \begin{eqnarray}
	p_{k} = a_{k} + b_{k}      
      \end{eqnarray}
    \item For example, when adding the numbers 
      \begin{eqnarray*}
        A  & = & 0.1010 \\
        + B  & = & 0.0011
      \end{eqnarray*}
      carries are generated in position 1 and propagated in positions
      0 and 3. 
    \end{enumerate}
  \item Carry Lookahead Equations. 
    \begin{enumerate}
    \item A carry enters position $k+1$ if a carry is generated in 
      position $k$ or a carry enters position $k$ and is propagated. 
      \begin{eqnarray*}
	c_{k+1} = g_{k} + p_{k} \cdot c_{k} 
      \end{eqnarray*}
    \item Solving this equation recursively gives:
      \begin{eqnarray*}
	c_{k+2} & = & g_{k+1} + p_{k+1} \cdot c_{k+1} \\
	& = & g_{k+1} + p_{k+1} \cdot (g_{k} + p_{k}
	\cdot c_{k}) \\
	& = & g_{k+1} + p_{k+1} \cdot g_{k} + p_{k+1}
	\cdot p_{k} \cdot c_{k}
      \end{eqnarray*}
    \item Extending this to one more position gives:
      \begin{eqnarray*}
	c_{k+3} & = & g_{k+2} + p_{k+2} \cdot c_{k+2} \\
	& = & g_{k+2} + p_{k+2} \cdot \\
        & & (g_{k+1} + p_{k+1} \cdot g_{k} + \
	p_{k+1} \cdot p_{k} \cdot c_{k}) \\
	& = & g_{k+2} + p_{k+2} \cdot g_{k+1} + \\
        & & p_{k+2} \cdot p_{k+1} \cdot g_{k} + \\
        & & p_{k+2} \cdot p_{k+1} \cdot p_{k} \cdot c_{k} 
      \end{eqnarray*}
    \end{enumerate}
  \item Generalized Carry Lookahead Equations. 
    \begin{enumerate}
    \item The equation for the carry into position $k+r$ is:
      \begin{eqnarray*}
	c_{k+r} & = & \left(\sum_{i=k}^{k+r-1} g_{i}
	\prod_{j=i+1}^{k+r-1}p_{j}\right) \
	+ c_{k} \prod_{j=k}^{k+r-1}p_{j} 
      \end{eqnarray*}
    \item Implementing this equation requires $r+1$ gates, which
      have a maximum 
      fan-in of $r+1$. 
    \item To produce all the carries from $c_{k+1}$ to $c_{k+r}$ a total of 
      \begin{eqnarray*}
	\sum_{i=1}^{r} (i + 1) = \frac{(r + 3) \cdot r}{2}
      \end{eqnarray*}
      gates are required.
    \end{enumerate}
  \item Modification of the original 9 gate FA. 
    \begin{enumerate}
    \item The 9 gate full adder already has a the required logic
      to produce $g_{k}$ and $p_{k}$. 
    \item In addition, the OR gate used to produce $c_{k+1}$ is no
      longer required. 
    \item Consequently, we can use a reduced full adder (RFA) that
      requires only 8 logic gates. 
    \item The generate and propagate signals are ready after 
      $1 \bigtriangleup$
    \end{enumerate}
  \item A $4$-bit CLA. 
    \begin{enumerate}
    \item For $k=0$ we have:
      \begin{eqnarray*}
	c_{1} & = & g_{0} + p_{0} \cdot c_{0} \\
	c_{2} & = & g_{1} + p_{1} \cdot g_{1} + p_{1} \cdot p_{0}
	\cdot c_{0} \\
	c_{3} & = & g_{2} + p_{2} \cdot g_{1} + p_{2} \cdot p_{1}
	\cdot g_{0} + \\
        & & p_{2} \cdot p_{1} \cdot p_{0} \cdot c_{0} 
      \end{eqnarray*}
      The logic used to produce the carries is typically
      referred to as a carry lookahead generator (CLG). 
    \item Draw figure of a 4-bit CLA. 
    \item A 4-bit CLG uses 9 gates and has two gate delays. 
      However, some of these gates have more than two inputs.
    \item Since a 4-bit CLA uses 4 RFAs, and a 4-bit CLG (9 gates), it has 
      a total of $4 \cdot 8 + 9 = 41$ gates. 
    \item The worst case delay for a 4-bit CLA is:
      \begin{eqnarray*}
	a_{k}, b_{k} \rightarrow s_{3} & = & 6 \bigtriangleup \\
	a_{k}, b_{k} \rightarrow c_{4} & = & 5 \bigtriangleup \\
      \end{eqnarray*}
    \end{enumerate}
  \item Block Carry Lookahead Generators
    \begin{enumerate}
    \item Each additional position, increases the fan-in of the logic gates. 
      Most technologies have a maximum fan-in of four. Therefore, to continue 
      the process, it is necessary to define generate and propagate signals 
      over four bit blocks. 
      \begin{eqnarray*}
	g_{k+3:k} & = & g_{k+3} + p_{k+3} \cdot g_{k+2} + \\
        & & p_{k+3} \cdot p_{k+2} \cdot g_{k+1} + \\
	& & p_{k+3} \cdot p_{k+2} \cdot p_{k+1} \cdot g_{k} \\
	p_{k+3:k} & = & p_{k+3} \cdot p_{k+2} \cdot p_{k+1} \cdot p_{k} 
      \end{eqnarray*}
    \item A four bit block carry lookahead generator (BCLG) has
      14 gates, and a maximum delay of $2 \bigtriangleup$. 
    \item In terms of the four bit block generate and propagate
      signals, the next carry can be expressed as 
      \begin{eqnarray*}
	c_{k+4} & = & g_{k+3:k} + p_{k+3:k} \cdot c_{k}
      \end{eqnarray*}
    \item A $4-bit$ CLA with block generate and propagate signals requires 
      46 gates. 
    \end{enumerate}
  \item Larger Carry Lookahead Adders
    \begin{enumerate}
    \item Draw 16-bit CLA. 
    \item A $16$-bit CLA requires $16$ RFAs ($8$ gates each) and
      $5$ CLA blocks ($14$ gates each). Therefore, it 
      requires a total of $16 \cdot 8 + 5 \cdot 14 = 198$ 
      gates.
    \item The delays through the 16-bit CLA are:
      \begin{eqnarray*}
	a_{k}, b_{k} \rightarrow p_{k}, g_{k} = 1 \bigtriangleup \\
	p_{k}, g_{k} \rightarrow g_{k+3:k} = 2 \bigtriangleup \\
	g_{k+3:k} \rightarrow c_{k+4} = 2 \bigtriangleup \\
	c_{k+4} \rightarrow c_{k+7} = 2 \bigtriangleup \\
	c_{k+7} \rightarrow s_{k+7} = 3 \bigtriangleup \\
	a_{k}, b_{k} \rightarrow c_{k+7} = 10 \bigtriangleup \\
      \end{eqnarray*}
    \end{enumerate}
  \item CLA Gate Counts 
    \begin{enumerate}
    \item An $n$-bit CLA with a maximum fan-in of $r$, requires 
      \begin{displaymath}
	\sum_{l=1}^{log_{r}(n)} \frac{n}{r^{l}} \approx \lceil
	\frac{n-1}{r-1} \rceil 
      \end{displaymath}
      carry lookahead blocks and $n$ RFAs. 
    \item An $r$-bit carry lookahead block requires $\frac{(3 + r)r}{2}$
      gates and each RFA requires $8$ gates. 
    \item Thus, the total number of gates for an $n$-bit CLA is 
      \begin{displaymath}
        8n + \frac{(3 + r)r}{2} \cdot \lceil \frac{n-1}{r-1} \rceil
      \end{displaymath}
    \item When $r = 4$, this gives
      \begin{displaymath}
        8n + 14 \cdot \lceil \frac{n-1}{3} \rceil \approx 12.67n - 4.67
      \end{displaymath}
    \item When $r = 2$, this gives
      \begin{displaymath}
        8n + 5(n - 1) = 13n - 5
      \end{displaymath}
    \end{enumerate}
  \item CLA Delay
    \begin{enumerate}
    \item An $n$-bit CLA with a maximum fan-in of $r$, requires
      $\lceil \log_{r}(n) \rceil$
      CLA logic levels. 
    \item An $r$-bit CLA has six gate delays and there are four
      additional gate 
      delays per level after the first. 
    \item Thus, the delay for an $n$-bit CLAs is
      \begin{displaymath}
        6 + 4 \cdot (\lceil \log_{r}(n) \rceil - 1) = \
	2 + 4 \cdot \lceil \log_{r}(n) \rceil
      \end{displaymath}
    \end{enumerate}
  \item Compared to the RCA, the CLA:
    \begin{enumerate}
    \item Has a much shorter delay (logarithmic instead of linear).
    \item Uses a larger number of gates.
    \item Is less regular in structure. 
    \end{enumerate}
  \item Carry Lookahead Adder Block Issues
    \begin{enumerate}
    \item Fan-out
    \item Logic Levels
    \item Wire Tracks (density)
    \end{enumerate}
  \item Quantitative Comparison
    \begin{table} [ht]
      \centering
      \label{cla1.tbl}
      \begin{tabular}{|r|r|r|r|r|} \hline 
        Size & CLA      & CLA      & RCA   \\  \hline
        $n$ &  ($r=4$) & ($r=2$)  &       \\ \hline \hline
        4  &  46      &  47      & 36    \\ \hline 
        8  &  92      &  99      & 72    \\ \hline
        16  &  198     &  203     & 144   \\ \hline 
        32  &  396     &  411     & 288   \\ \hline 
        64  &  806     &  827     & 576   \\ \hline 
        128 &  1,612   &  1,659   & 1,152 \\ \hline
      \end{tabular}
      \caption{Area}
    \end{table}
    \begin{table} [ht]
      \centering
      \label{cla2.tbl}
      \begin{tabular}{|r|r|r|r|r|} \hline 
        Size & CLA     &  CLA    & RCA \\ \hline
        $n$  & ($r=4$) & ($r=2$) &     \\ \hline \hline
        4   &   6     &  10     &  12 \\ \hline 
        8   &   10    &  14     &  20 \\ \hline
        16   &   10    &  18     &  36 \\ \hline 
        32   &   14    &  22     &  68 \\ \hline 
        64   &   14    &  26     & 132 \\ \hline 
        128  &   18    &  30     & 260 \\ \hline   
      \end{tabular}
      \caption{Delay}
    \end{table}
  \end{enumerate}
  \begin{figure*}
    \label{cla16.fig}
    \begin{center}
      \setlength{\unitlength}{0.0105in}%
      \epsfig{figure=cla16.eps, height=6.0in}
    \end{center}
    \caption{16-bit Carry Lookahead Adder.}
  \end{figure*}
\end{enumerate}



\end{document}
