\documentclass[times, 10pt, twocolumn]{IEEEtran}
\usepackage{epic}
\usepackage{hyperref}
\usepackage{latexsym}
\usepackage{pifont}
\usepackage{psfrag}
\usepackage[]{fontenc}
\usepackage[latin1]{inputenc}
\usepackage{array}
\usepackage{color}
\usepackage{graphicx}
%\usepackage{simplemargins}
\usepackage{epsfig}
%\usepackage{fancyheadings}
%\usepackage{lastpage}
%\pagestyle{fancy}
%\lhead{High Speed Computer Arithmetic}
%\rhead{\today}
%\cfoot{Page \thepage\ of \pageref{LastPage}}
%\setallmargins{1.0in}

\begin{document}
\title{High Speed Computer Arithmetic\\
ECEN 4233 Final Project (v1.0.3)\\
Spring 2024}
\author{James E. Stine\\
Electrical and Computer Engineering Department\\
Oklahoma State University\\
Stillwater, OK 74078, USA}
\date{}
\vskip 0.5cm

\maketitle
%\thispagestyle{plain}\pagestyle{plain}

The project~\footnote{Note that this
document is updated periodically to provide you with the
most reliable information to complete the project.  Although
previous versions of this document are basically correct, frequent
revisions are made in order to help you save time and {\bf not}
to hinder your performance.  Therefore, make sure you
check back periodically for updates.  I will add revision
numbers to help you identify which version of this document
you have.}  in this course is designed to test your knowledge of the topics
discussed in lecture and apply them to a real-world problem.  The project
will consist of a task and you should work individually and submit one
set of files.  Under no circumstances will copying be allowed on this project and
any piece of software, hardware, or any other item that is copied may result
in a failing grade for this class.  However, I do encourage that discussion
should be exhibited between all individuals to understand the problem.  If you
need any clarification on what constitute an illegal action, please feel free
to contact me.

In this project, you will be implementing an efficient method for
approximating division using Goldschmidt's algorithm or
sometimes called division by convergence. For graduate students,
this project will involve modifying your implementation, so
that it can also produce the square root 
of your input operand.  The
Goldschmidt iteration is an attractive method because it converges
quadratically.  This means that one iteration will double the number
of accurate bits in the approximation.  For example, if I have an
approximation that is accurate up to $2$ bits, after the next
iteration, it will be accurate up to $4$ bits.  The Goldschmidt
algorithm can be applied to a variety of different systems and is
also attractive for square-root thus making it popular for modern
high-performance processors.

You will also implement this structure for IEEE 754 floating-point and
support both round-to-nearest even and round-to-zero rounding modes~\cite{30711}.
This will involve computing the remainder as well as ancillary logic
to handle the exponent.

\section{Basic Baseline Project}

This project contains a significant amount of work which will require
substantial design effort.  Therefore, it is probably a good idea to start
early.  For this project, you will be following the details listed below:
\begin{itemize}
\item A $32$-bit IEEE 754 Divide (and possibly Square-Root) Unit that uses Goldschmidt's
  iteration to compute its operation.  The unit is designed for use in
  a single-precision floating-point unit, so it must be accurate to
  $23$ fractional bits.  That is, one ($1$) integer and twenty-three
  ($23$) fractional-bits.
\item You will need to add the remainder to properly handle the
  rounding mode.  
\item This is an individualized project but you are welcome to work
  together but you should each submit your final HDL.
\item In order to verify your results, you must use Java or C (or some
  other high-level language) to
  verify your design constraints.  Sample Java and C code is provided
  to help you start this process, as explained in class.
\item The HDL is due {\bf April 26, 2024} at midnight
  with no exceptions except approved medical and legitimate excuses.
\item The design should be implemented and tested with SystemVerilog.
  Several design vectors should be tested to make sure it works appropriately.
\item Table~\ref{table1.tbl} lists some \textit{suggested}
  input/output signals.  Please feel
  free to add more or delete some, if you wish.  Sizes and names are
  only suggested and may be modified accordingly.
\item You should only employ \underline{one ($1$)} 
  multiplier in
  your design, therefore, your final result may take several cycles.
  Your design can have any number of adders, counters, or compressors,
  but you may only use {\bf one} multiplier in this project.  
\end{itemize}
\begin{table*} [ht!]
  \centering
  \begin{tabular}{|c|c|c|l|} \hline
    $Signal$ & Input/Output & Size (bits) & Description \\ \hline \hline
    N         & Input  & $32$     & Input operand (dividend) \\ \hline
    D         & Input  & $32$     & Input operand (divisor) \\ \hline
    Operation & Input  & variable & Operation selector \\ \hline
    rm        & Input  & $2$      & Rounding mode \\ \hline
    clk       & Input  & $1$      & Clock  \\ \hline
    Load      & Input  & $1$      & Load registers \\ \hline
    Q         & Output & $32$     & Output operand \\ \hline
  \end{tabular}
  \caption{Suggested Pins and Operand Sizes}
  \label{table1.tbl}
\end{table*}

\section{Overall Goal}

This project does require a logic simulation using
ModelSim/QuestaSim, as described
in our examples with class and our repository.
For this project, you are expected to demonstrate your
understanding of the
procedures needed to compute the final result, an adequate error analysis,
and some simulation to show that your procedure would work.  One method to
complete an acceptable 
simulation is to write a Java or C program to determine the necessary
bit sizes to meet a given accuracy (in our case $23$ bits).  Another
method might use a
programming environments like MATLAB or Maple.  The best one is to use one that
you feel comfortable with.  However, based on previous semesters where
students did not understand the importance of this process and fudged
Excel, you must
use a program to verify the accuracy of your final computation.
Therefore, you can use the provided Java or C programs to verify the
sizes from each unit.  Although you are welcome to use another method,
such as from MATLAB, it must be something equivalent to the provided C
or Java or code in its operation.

For any algorithm and especially for algorithms that may be employed
to save or protect
people's lives, it is imperative that an
analysis of the error involved in the computation be completed.  That is,
a worse-case scenario should be examined so that the precision required is
satisfactory met.  In other words, that we can compute an operation between
two $24$-bit operands and the answer should be accurate to less than a unit
in the last position (ulp) of the precision required (i.e. $23$ fractional bits).

\subsection{Required Elements}

The following are required elements to the project:
\begin{itemize}
\item Name your top-level design \verb!fpdiv! or \verb!fpdivsqrt!.
\item A program or mathematical analysis showing your algorithm will
  work.  Usually, designers utilize a program to verify your
  operation, as well as your testbench, that
  you feel comfortable with.  You {\bf must} use the C or Java program
  (or equivalent) to test your design.  It is also advisable to test
  several hundred vectors to see if it meets your requirements.  This
  should be relatively easy within the program.  Ask me, if you need
  help with this.  
\item A block diagram of the hardware necessary to implement your algorithm.
  It is easiest to take a hierarchical approach to documenting your block
  diagram.   That is, display the top-level block diagram on one page, and
  then on each subsequent page show what each lower-level block diagram is
  composed of.  Diagrams usually take a ridiculous amount of time to create
  and an early implementation will help you complete this project effectively.
  Although I have done many figures over my lifetime, I am still looking for
  the best program to create figures.  I would highly recommend a program
  like Microsoft Visio, draw.io, or xfig.  Please try to simplify the
  diagrams as they can be ever consuming and completing them can
  really be a time sink.
\item You should use
  HLS (i.e., \verb!assign Z=A*B!
  to generate the multiplier to get thigns started, however, you are welcome to use the
  Wallace multiplier for efficiency found within the repository.
\item ECEN 5080 students are also required to implement floating-point
  square root in addition to floating-point division.
  ECEN 4233 students just have to implement floating-point
  division.  
\item I do expect you to document all combinational and sequential logic you
  have in your design, but try to summarize the detail of your
  design.
\item Your final design must pass all the TestFloat~\cite{hauser}
  validation values (see the HSCA repository).
\item You may use any hardware element discussed in class provided you
  discuss it completely in your project.  For this year, let's try to
  summarize as much as you can to make it succinct and easy to
  produce.  I am putting more emphasis this year on the
  implementation.
\item You do not have to worry about IEEE 754 exceptions and can be
  added on a case-by-case basis for extra credit (e.g., $\pm\infty$).  
\end{itemize}

\subsection{Steps}

Sometimes, I get asked what steps are necessary in this project to
meet the given constraints and
although I encourage adaptability and individuality, the question is
valid and something I should address.  These are the suggested steps -
feel free to improvise or skip steps, if necessary.  There are no
right or wrong ways to complete the project.  It is important to
emphasize that this process is what digital designers typically
encounter when implementing a unit at companies like Intel, AMD, ARM,
NVIDIA, and others.  Users demand accountability that any unit you
utilize will function, as expected.  Therefore, it is a good thing to
learn the process you utilize in this project.
\begin{enumerate}
\item The main repository for all the files is available here:
  \url{https://github.com/stineje/HSCAProjectS24/}.    
\item Inspect the C or Java code and learn how to compile and run it
  repeatedly.
\item Modify the C or Java code to determine the sizes of your units
  inside your main functional unit.  Make sure you choose a data type
  for your results that satisfy the given input and output precision.
\item Run an adequate number of vectors to make sure your design meets the
  accuracy for the input and output operands.  Save these vectors, so
  you can utilize them to test your SystemVerilog test
  benches.  
\item Put together a block diagram of your design with the sizes you
  determined in previous steps.
\item Stub out your SystemVerilog along with a testbench for your
  design.  Make sure your testbench works!
\item Once you have a reasonable block diagram, start implementing
  lower-level blocks in SystemVerilog where you try to use
  these lower-level blocks to validate their functionality.  It helps
  to fully test lower-level blocks as these blocks are
  sometimes the most problematic within a design.  Also, its easier to
  exhaustively test a lower-level design, as it usually has a low
  number of input and output ports.
\item Start assembling the upper-level SystemVerilog 
  blocks and testing whether they match the results from your C or
  Java code line by line.  
  It is advisable to start out with a simple design (e.g.,
  only a multiplier) and work upwards.  Always verify the operation of
  your design before starting on working a new block.
\item Finalize your final SystemVerilog block and testbench.  You may
  have several small iterations here to figure out the input and
  output signaling as well as handling the control structure.
\item For the two's complementator part of Goldschmidt operation, I would
  recommend using the one's complement for efficiency and optimization
  as mentioned in class.
\end{enumerate}

\subsection{Rounding}

Rounding is not easy for multiplicative divide.  It is even more
difficult to describe, but the main idea is that we add an extra
constant to the quotient and use the remainder to see how far we are
from the rounding.  Fortunately, I am only asking you to perform two
rounding functions: round-to-nearest-even (RNE) and round-to-zero
(RZ).  We will
use the techniques published here~\cite{540618, 762835}.  This process
amounts to rounding to one additional bit or $pc+1$ bits.  This will
provide the necessary mechanism to produce a
result that has an error no worse than $\pm0.5$~ulp, or $2^{-pc}$.

The rounding function assumes that a biased trial result $q_i$ has been
computed with $pc + 1$ bits, which is known to have an error of
$(-0.5,+0.5)$~ulp with respect to $pc$ bits. The extra bit, or guard bit,
is used along with the sign of the remainder, a bit stating whether
the remainder is exactly zero, and the final result is chosen such that
it has three possible results, either $q$, $q-1$, or $q+1$.
Again, for RNE and RZ this will be trivial.  The rounding details are
shown in Table~\ref{round.tbl}
\begin{table} [b!]
  \centering
  \begin{tabular}{|c|c|c|c|c|c|c|}\hline
    Guard Bit & Remainder & RN & RZ \\ \hline \hline
    0     & $=0$   & trunc & trunc \\ \hline
    0     & $-$    & trunc & dec   \\ \hline
    0     & $+$    & trunc & trunc \\ \hline
    1     & $=0$   & RNE   & trunc \\ \hline
    1     & $-$    & trunc & trunc \\ \hline
    1     & $+$    & inc   & trunc \\ \hline
  \end{tabular}
  \caption{Action for Round Function}
  \label{round.tbl}
\end{table}

This rounding is complicated and you just have to make sure you add
the right constant to make sure the rounding mode is accurate.
Technically,  this forces a $1$ into the Least Significant Bit (LSB) to
guarantee the precision needed~\cite{762835}.  This
amounts to the following additions for $q$, $q-1$, and $q+1$ for $q$,
$qp$, and $qm$, respectively (assuming a $32$-bit multiply -- you can
change to reduce area of your multiplier).
\begin{center}
\begin{verbatim}
assign q_const = {32'h0000_0040};
assign qp_const = {32'h0000_0140};
assign qm_const = {32'hFFFF_FF3F};
\end{verbatim}
\end{center}
You  need to store these three values 
during the final iteration and choose the correct value based on
Table~\ref{round.tbl}.  These outputs are not necessary during the
normal Newton-Raphson iteration.

What is interesting about Table~\ref{round.tbl}, is the RNE case in
the middle of the table.
It should be noted that for division where the precision of the
input operands is the same, which is true in our case and for most
IEEE 754 units, the exact halfway
case cannot occur.  Therefore, you do not have to check this option
which makes things abundantly easier.

To compute the remainder during the rounding phase, you can reverse
the table to make things easier.  Some researchers call this process
back multiplication~\cite{762835}.  That is, instead of performing the operation
\begin{eqnarray*}
  b \times q - a \enspace ,
\end{eqnarray*}
is implemented instead as:
\begin{eqnarray*}
  a - b \times q \enspace .
\end{eqnarray*}
That is, the sign of this remainder is thus the negative of the true
remainder sign.  The remainder can get tricky because of the range of
the quotient.  For example, the input operands should be $[1,2)$,
however, the output can be in the following range $[0.5, 2)$.  This
range will affect your remainder computation and should be
accounted for.  




\section{Grading}

Every year I get asked on the grading that I utilize to ascertain what grade
  you get for this project.  My goal is to also teach you, besides the theory,
  that verbal communication is vital to your success.  I suggest that you
  spend at least one week for the write-up with this project.  I am a fair
  grader, but I have certain biases, so please make note of the following
  items:
\begin{itemize}
\item  Please note that there is no correlation between how much work
  you complete and the grade of your project. I have had students in the past
  who have worked tirelessly on the project, but ended up with a C, because
  they failed to document anything.  That is, they had a wonderful design,
  but gave up with the design and handed in an inaccurate picture of their
  work.  Therefore, I encourage you to start early and commit regular
  updates to this project.
\item I am here to help you, but I cannot spend time debugging your
  project.  I would highly advise working from the ground up and assembling
  parts after you test them.  Always test your output against a known
  value, preferrably from the C code.
\item Your project will be based on the following break down:
\begin{itemize}
\item $70\%$ Bi-Weekly status reports (see below in Section~\ref{Status.sec}).
\item $20\%$ Final functional block diagram and SystemVerilog code.
\item $10\%$ Testing : final testing against TestFloat (see below).
\end{itemize}
\item A working SystemVerilog model of your design showing
  several vectors accurately computing your answer (including a final test
  bench and DO files).  There is no need to show me all the lower
  level test benches - a final test bench is sufficient.  All files
  \textbf{must} be handed in with your project
  as one compressed file (e.g., zip), so it may be verified for operation.
\item Please hand in all HDL to Canvas by the deadline.  There are no
  exceptions to this other than medical and legitimate excuses.  
\end{itemize}

\section{Meetings/Bi-Weekly Status Report}
\label{Status.sec}

I have had excessive problems with a small number of students in this
class in that they wait until
the last week to implement anything (Note: not all students do this
but I would say about half abuse the deadline).  I have tried multiple ideas
including suggested milestones, however, nothing has worked.
Then, I get a barrage of Emails
similar to ``things are not working and I am not sure why?''.
I also get a flurry of ``I wish you did something to keep us on track!''.  My
thought process here is to help you with this option and this is the third
time I am trying this (the first/second time was not as good as I had hoped,
but I am hoping that the third time is the golden ticket).
For this reason, I am forgoing the report to have bi-weekly status reports.
In summary, I am
using the suggestion from a colleague/friend to have weekly status
reports instead which has worked for him.  

You will be assessed on this bi-weekly status report.
I will require that you have a
one page summary on what you have worked on, what works, what you have done.
You can bring diagrams or show me anything you have done and I can
assess your grade weekly.
If you skip a week, I will give you $0$ points for your grade that
week.  I would
suggest visiting with me during office hours, but you can visit with me
anytime that we are both free.  However, you must visit with
me each week to get a grade; otherwise you will get a $0$ for that
week.
I will \underline{not} conduct the meeting over 
Zoom unless you are a remote student
or have legitimate problems attending in person.
You \textbf{must} meet with me
weekly to get full credit.  Remote students
should contact me to arrange a weekly Zoom or Teams
discussion.  These meetings must be completed by Friday at $5$ PM each week
starting the week of Monday, April 1, 2024.  This amounts to about $2$
mandatory meetings for the semester, however, you are welcome to visit with me
more than $2$ times during the semester.  I also suggest that you do not
try to visit with me last minute (e.g., 4:45 PM on a Friday).

Please note that a significant amount of your grade is based on this
meeting.  I would suggest that you be be prepared
for each meeting with the $1$ page status report you hand into
me.  I will give you a grade at the end of our meeting
so you have some feedback if you are underperforming or are on track
for completing the design.  Again, I would highly suggest you prepare
for our meeting and it should last about $5$-$15$ minutes at maximum.
If you are not prepared, your grade will represent this.

\section{Hints}

The best way to complete this project is to work from the ground up.  That
is, build the version first for a small size (e.g. $16$ bits) or a
simple design by adding more functionality.
%This
%is typically called bit-slicing and even though we are not laying down
%silicon structures, digital designers utilize this modularity within their
%hierarchy to create efficient designs.
Hardware designers sometimes make mistakes
by initially trying to put too much into a hardware design.  It is advisable
to make the logic simple and expand the logic incrementally.

I would highly advise creating all the combinational logic
\underline{first} before you go on to the sequential logic, if
necessary.  Sequential
logic can change due to the large number of signals that may need to be
controlled.  Therefore, a three step process is recommended:
\begin{itemize}
\item Draw the complete block diagram of all combinational elements on a nice
  large piece of paper.  Some students have commented that legal paper makes
  a good way of designing logic blocks.
\item  The baseline project will be discussed rather
  regularly in class, therefore, paying attending to the discussion on
  different days may clear up any confusion on its implementation.
\item You are welcome to use structural, RTL or HLS-level coding
  for your SystemVerilog.  
\item Make a block diagram of your control unit.  Your control logic will
  probably be synchronous-based, therefore, it is probably a good idea
  to have the complete control tabulated for each cycle. You do
  \textbf{not} have the
  control implemented, but it should be integrated within your
  testbench.  That is, you
  can hard code the control logic into your testbench and test whether your
  datapath will work.  Some students have implented the control unit
  to make things easier, but the choice is yours. 
\item I will provide a set of tests that you should use using the
  University of California-Berkeley (UCB) TestFloat test suite~\cite{hauser}.  You  
  will need to have your testbench read these values in and test
  against  your unit (a sample testbench will be provided).
\item There are many ways to get extra credit in this project by implementing
  extra features.  However, every year I have students spend way too much
  time trying to get extra credit working and not enough time on the
  documentation.  I would highly suggest starting the
  baseline project and
  co-implementing the extra credit to allow you time to complete everything
  in an orderly fashion.  
\end{itemize}

\bibliographystyle{IEEEbib}
\bibliography{projectS24}


\end{document}
